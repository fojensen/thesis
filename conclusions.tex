\chapter{Conclusions}
\label{chap:conclusions}

This thesis has attempted to give a broad overview of the field of high-energy and collider-based particle physics. A brief introduction to the Standard Model was given in Chapter 2. This is a quantum field theory describing the fundamental particles and gauge-mediated interactions between them which govern the matter of our Universe; these particles were summarized in Figure \ref{fig:sm}. In Chapter 3, we gave an overview of Supersymmetry, just one of many possible extensions to the SM which may in fact be realized in Nature. In Chapter 4 we gave a description of the Large Hadron Collider, the 27$\,$km machine which houses a ring of two separate proton beams and provides the source of the high-energy proton-proton collisions. In Chapter 5 we gave a description of the CMS detector, an onion-like apparatus consisting of multiple layers of different particle detector technologies. Chapter 6 demonstrates how the hit information and data from the detector allows us to reconstruct the particles produced in the final states of these collisions. The SM is thus far able to provide a very good model of what we observe in these sorts of experiments.

We concluded in Chapter 7 of how an analysis of data collected by CMS is able to spot signs of physics beyond the Standard Model:

A search for physics beyond the SM was presented using events with boosted H bosons and missing transverse energy (\ptmiss). The search targeted events with two or more wide-angle jets (AK8) being consistent with the decay of a boosted H or Z boson decaying to \bbbar. \ptmiss could potentially arise in the case a supersymmetric particle escapes detection. An ABCD method uses a sideband region to predict the the SM background in our signal region. Events are categorized according to the \bbbar and mass tagging of the leading two AK8 jets in the event. The observed yields in the 6 signal bins are statistically compatible with the SM background expectation and no excess of events is observed. We use these results to set limits on the gluino mass for the SUSY-inspired T5HH or T5ZH models. For the T5HH model we are able to exclude gluino masses below 2010 GeV at 95\% confidence level. This is with the assumption the NLSP mass is 50 GeV less than the gluino mass and that the LSP has a mass of 1 GeV. The work presented here has been published in Phys. Rev. Lett. \cite{CMS-SUS-17-006}.
