\chapter{Introduction}

The emerging picture of the last century is that the elements described by the Periodic Table are themselves not the most fundamental forms found in Nature. We have learned that these elements are made of atoms consisting of bound states of protons, neutrons, and electrons. We have discovered that these protons and neutrons themselves are made of more fundamental components called quarks and gluons. And so as the Periodic Table before it, the Standard Model of particle physics (SM) seen in Figure~\ref{fig:sm}, has been developed to provide an organizing principle for what we currently understand to be the correct description of the fundamental particles and forces which form the Universe. The discovery of the Higgs boson in 2014 \cite{higgsdisc} completed the search for all the fundamental particles addressed within the theory, a truly monumental achievement. As time goes on, its experimental verification only grows stronger.

\begin{figure}
\centering
\includegraphics[width=0.6\textwidth]{figs/StandardModelofElementaryParticles.pdf}
\caption{The Standard Model of Particle Physics.}
\label{fig:sm}
\end{figure}

But, there are many both experimental and theoretical indications that the SM is not the final story. Cosmological observations hint at the presence of a ``dark matter'' in the Universe, a ubiquitous substance thought to account for 85\% of the total matter budget in the Universe. Although its influence has been observed in gravitational lensing phenomena and galaxy rotation curves, we do not yet have a particle description of its nature. One theoretical motivation which points toward the SM being a part of some grander theory is known as a ``fine-tuning'' problem. Because the Higgs boson is a scalar particle (the only such fundamental in the SM), its mass receives quantum mechanical corrections which are strongly dependent on the ultraviolet cutoff $\Lambda_{\textrm{UV}}$ and other high mass scales in the theory. As the most natural cutoff is likely at the GUT or Planck scale, we would expect the Higgs boson mass to be extremely large, many orders of magnitude than it has experimentally been determined to be. This implies that there is some sort of ``un-natural'' collusion between the terms, which is somehow able to bring the mass to the electroweak scale.

Supersymmetry (SUSY) is an elegant extension to the SM which is able to address many of these issues by the introduction of a new symmetry of the Lagrangian which relates fermionic and bosonic states. The particle content of SUSY is over twice that of the SM as the theory requires a doubling of the particle content to give each of the SM particles a ``partner'' of spin differing by 1/2 unit. Among its phenomenology, the lightest particles of the theory, expected to remain stable, are able to serve as candidates for dark matter. The addition of these superpartners to the theory yield quantum mechanical corrections to the Higgs mass which enter with the opposite sign naturally are able to cancel the terms of high-order in $\Lambda_{\textrm{UV}}$. An exact supersymmetry requires that these superpartners have identical mass to their SM counterpart. This is obviously not the case and therefore, if SUSY is correct, these new particles must have acquired mass through some other means. These masses must be sufficiently large such that their production at our collider experiments has not allowed for their unambiguous detection. A major motivating force for the construction and experimentation at the LHC is that of finding evidence of SUSY. The analysis presented in this thesis presents one such search, focusing in one small parameter space of the vast possibilities.

The thesis is outlined as follows: A description of the Standard Model (SM) of particle physics, the theory describing the known fundamental forces and particles, is presented in Chapter~\ref{chap:sm}. A description of the Minimal Supersymmetric Standard Model (MSSM), one extension to the SM able to provide answers to many of our questions, is presented in Chapter~\ref{chap:mssm}. A description of the Large Hadron Collider (LHC), the facility which acts as our source of high-energy proton-proton collisions, is presented in Chapter~\ref{chap:lhc}. A description of the CMS particle detector, responsible for the detection of particles and reconstruction of the proton interactions, is presented in Chapter~\ref{chap:detector}. A description of how the data from the detector are reconstructed and identified as physical particles is presented in Chapter~\ref{chap:eventreco}. The focus of the thesis, a description of the how the physics data can be used to search for evidence of new particles such as those predicted by SUSY, is presented in Chapter~\ref{chap:analysis}. The conclusions are presented in Chapter~\ref{chap:conclusions}. Appendix~\ref{chap:bb} presents a more detailed discussion of the \bbbar-tagging algorithm used in this analysis. Appendix~\ref{chap:reinterpretation} provides resources for additional physics interpretations of our results presented in Section \ref{chap:analysis}. Appendix~\ref{chap:bbsf} presents the calculation of data-mc scale factors for $b\bar{b}$-tagging W boson jets, a partly independent topic from the rest of the thesis.
